\documentclass[12pt]{article}

\usepackage{ebproof}
\usepackage{tikz-cd}
\usepackage[a4paper,margin=15mm]{geometry} 
\usepackage{amsmath,amsthm,amssymb}

\newcommand\A{\varphi}
\newcommand\B{\psi}
\newcommand\CC{\chi}
\newcommand\DD{\sigma}
\newcommand\E{\tau}
\newcommand\GG\Gamma
\newcommand\D\Delta
\newcommand\T\Theta
\newcommand\LL\Lambda
\newcommand\SM\Sigma
\newcommand\PP\Pi
\newcommand\TO\Rightarrow
\newcommand\PC[1]{\mathbf{#1}}
\newcommand\NC\square
\newcommand\PB{\lozenge}
\newcommand\FV[1]{\mathrm{FV}({#1})}
\newcommand\DEG[1]{\mathrm{deg}({#1})}
\newcommand\RK[1]{\mathrm{rank}({#1})}
\newcommand\R[1]{\mathrm{R^{#1}}}

\newcommand\AX{\textrm{Ax}}
\newcommand\LB{\textrm{L$\bot$}}
\newcommand\LW{\textrm{LW}}
\newcommand\RW{\textrm{RW}}
\newcommand\LC{\textrm{LC}}
\newcommand\RC{\textrm{RC}}
\newcommand\LE{\textrm{LE}}
\newcommand\RE{\textrm{RE}}
\newcommand\LN{\textrm{L$\neg$}}
\newcommand\RN{\textrm{R$\neg$}}
\newcommand\LA{\textrm{L$\land$}}
\newcommand\RA{\textrm{R$\land$}}
\newcommand\LO{\textrm{L$\lor$}}
\newcommand\RO{\textrm{R$\lor$}}
\newcommand\LI{\textrm{L$\to$}}
\newcommand\RI{\textrm{R$\to$}}
\newcommand\LU{\textrm{L$\forall$}}
\newcommand\RU{\textrm{R$\forall$}}
\newcommand\LX{\textrm{L$\exists$}}
\newcommand\RX{\textrm{R$\exists$}}
\newcommand\Cut{\textrm{Cut}}


\newcommand\CI{\textrm{$\land$I}}
\newcommand\CE{\textrm{$\land$E}}
\newcommand\DI{\textrm{$\lor$I}}
\newcommand\DE{\textrm{$\lor$E}}
\newcommand\II{\textrm{$\to$I}}
\newcommand\IE{\textrm{$\to$E}}
\newcommand\UI{\textrm{$\forall$I}}
\newcommand\UE{\textrm{$\forall$E}}
\newcommand\XI{\textrm{$\exists$I}}
\newcommand\XE{\textrm{$\exists$E}}

\newcommand\EFQ{\bot_\mathbf{i}}
\newcommand\RAA{\bot_\mathbf{c}}
\newcommand\PLR{\mathrm{P}}

\newcommand\DT{\mathrm{DT}}
\newcommand\DTT{\DT^{-1}}
\newcommand\DTX[1]{\DT^{#1}}

\newcommand{\asm}[1]{\delims{[}{]^{#1}}}

\newcounter{thm}
\newtheorem{lemma}[thm]{Lemma}

\begin{document}

\title{HW9}
\author{Y. Konpaku}

\maketitle

\section{QUESTION}
\begin{enumerate}
    \item Prove the case of $(\LX)$ and $(\RX)$ for $\PC{G3[mi]}$ in $3.5.2(\alpha)$.
\end{enumerate}

\section{ANSWER}
\begin{enumerate}
    \item \begin{proof}\ 
        \begin{description}
            \item[$(\RX)$] For arbitrary fresh renaming $*$ free in the conclusion $\GG\TO\exists x\A$, we can find a proper renaming $**$ s.t. $\GG^{**}\equiv\GG^*$. If $*$ does not rename the bounded $x$ in $\exists x\A$, we let $**$ has the same activity on $\A$ as $*$. Then we can obtain that $(\A^x_t)^*\equiv(\A^x_t)^{**}\equiv(\A^{**})^x_t\equiv(\A^*)^x_t$ hence the effects of bounded variable renaming and term substitution are orthogonal. We can apply $(\RX)$ on $[\GG\TO\A^x_t]^{**}\equiv\GG^*\TO(\A^*)^x_t$ by induction hypothesis which concludes $\GG^*\TO\exists x\A^*\equiv[\GG\TO\exists x\A]^*$ since $\exists x\A^*\equiv(\exists x\A)^*$ for $*$ is free for $x$ in $\exists x\A$.
            
            Otherwise, $\exists x\A$ will be rename to $\exists y(\A^*)^x_y$ s.t. $y\not\in\FV{\A^*}=\FV{\A}$. Since $y$ is free in $\A$ we have $\A^x_t\equiv{\A^x_y}^y_t$. Then we can define $**$ as the same way and use the fact that substitution and renaming are commutable again: $[\GG\TO\A^x_t]^{**}\equiv\GG^*\TO{(\A^*)^x_y}^y_t$. If we apply $(\RX)$ on it by induction hypothesis then we can conclude $\GG^*\TO\exists y(\A^*)^x_y$, which is exactly what we want.
            \item[$(\LX)$] The case that $*$ does not rename $x$ in $\exists x\A$ in conclusion is very similar to $(\RX)$, now we focus on the case that $(\exists x\A)^*\equiv\exists z(\A^*)^x_z$ for some $z$ fresh. Since $z$ is fresh, we have $(\A^x_y)^*\equiv(\A^*)^x_y\equiv{(\A^*)^x_z}^z_y$. We can apply $(\LX)$ on $[\GG,\A^x_y\TO\B]^{**}\equiv\GG^*,{(\A^*)^x_z}^z_y\TO\B^*$ to obtain $\GG^*,\exists z(\A^*)^x_z\TO\B^*$ because $y$ is free for $\GG,\B$ and $\exists x\A$(this implies $y\equiv x$ or $y\notin\FV{\A}=\FV{(\A^*)^x_z}$) and changing bound variables does not affect free variables.
        \end{description}
    \end{proof}
\end{enumerate}

\end{document}